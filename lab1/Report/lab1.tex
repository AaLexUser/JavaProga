\documentclass[12pt,onecolumn]{article}

\usepackage{listings}
\usepackage{float}
\usepackage{mathtools}
\usepackage[russian]{babel}
\everymath{\displaystyle}
\usepackage{color}
\usepackage[usenames]{color}
\usepackage{geometry}
\geometry{
  a4paper,
  top=25mm, 
  right=15mm, 
  bottom=25mm, 
  left=15mm
}

\begin{document}

\begin{center}
    Санкт-Петербургский Национальный Исследовательский\\ 
    Университет ИТМО\\
    Факультет Программной Инженерии и Компьютерной Техники\\
    \includegraphics[scale=0.3]{itm.jpg} % нужно закинуть картинку логтипа в папку с отчетом
\end{center}
\vspace{1cm}


\begin{center}
    \large \textbf{Вариант №333893}\\
    \textbf{Лабораторная работа №1}\\
    по дисциплине\\
    \textbf{\textcolor{red}{\textit{'Программирование'}}}
\end{center}

\vspace{2cm}

\begin{flushright}
  Выполнил Студент  группы number\\
  \textbf{Student's name}\\
  Преподаватель: \\
  \textbf{Teacher's name}\\
\end{flushright}

\vspace{6cm}
\begin{center}
    г. Санкт-Петербург\\
    2021г.
\end{center}

\newpage
\section{Текст задания}
\begin{enumerate} 
\itemСоздать одномерный массив a типа short. Заполнить его чётными числами от 6 до 20 включительно в порядке возрастания.
\itemСоздать одномерный массив x типа float. Заполнить его 17-ю случайными числами в диапазоне от -3.0 до 15.0.
\itemСоздать двумерный массив a размером 8x17. Вычислить его элементы по следующей формуле (где x = x[j]):
\begin{itemize}  
\item если a[i] = 16, то a[i][j]=$e ^{(\frac{2}{3} + x \cdot (x-1))^3}$;
\item если $a[i] \in \{6, 12, 18, 20\}$, то $a[i][j]=\frac{1}{3}\cdot\ln((|x|+1)^x)$
\item для остальных значений a[i]: $a[i][j]=\frac{1}{2}/(3 - \arctan(\cos(x)))$.
\end{itemize}
\item Напечатать полученный в результате массив в формате с двумя знаками после запятой.
\end{enumerate}
\lstloadlanguages{Java}
\section{Исходный код программы}
\definecolor{mygray}{RGB}{120,120,120}
\lstset{extendedchars =/true,
breaklines=true,
basicstyle=\ttfamily\fontsize{9pt}{9pt}\selectfont,
commentstyle=\color{mygray},
keywordstyle=\color{blue}}
\lstinputlisting[language=Java,numbers=left]{../Main.java}
\newpage
\section{Результат выполнения}
\begin{lstlisting}[basicstyle=\ttfamily\fontsize{6pt}{6pt}\selectfont]
5,52	0,04	7,05	5,52	9,16	0,24	-0,12	-0,01	12,74	6,98	-0,29	1,98	8,92	1,76	0,12	12,00	-0,12	
0,18	0,22	0,13	0,18	0,17	0,20	0,21	0,23	0,17	0,13	0,19	0,14	0,16	0,13	0,21	0,20	0,21	
0,18	0,22	0,13	0,18	0,17	0,20	0,21	0,23	0,17	0,13	0,19	0,14	0,16	0,13	0,21	0,20	0,21	
5,52	0,04	7,05	5,52	9,16	0,24	-0,12	-0,01	12,74	6,98	-0,29	1,98	8,92	1,76	0,12	12,00	-0,12	
0,18	0,22	0,13	0,18	0,17	0,20	0,21	0,23	0,17	0,13	0,19	0,14	0,16	0,13	0,21	0,20	0,21	
Infin	1,08	Infin	Infin	Infin	1,39	420,99	1,75	Infin	Infin	Infin	Infin	Infin	Infin	1,10	Infin  625,07	
5,52	0,04	7,05	5,52	9,16	0,24	-0,12	-0,01	12,74	6,98	-0,29	1,98	8,92	1,76	0,12	12,00	-0,12	
5,52	0,04	7,05	5,52	9,16	0,24	-0,12	-0,01	12,74	6,98	-0,29	1,98	8,92	1,76	0,12	12,00	-0,12
\end{lstlisting}
\section{Вывод}
В этой лабораторной работе я научился работать с:
\begin{itemize}
\item Одномерными и двухмерными массивами;
\item Математическими функциями стандартной библиотеки Java;
\item Циклами и ветвлениями;
\item Форматированным выводом числовых данных;


\end{itemize}
\end{document}